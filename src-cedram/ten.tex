\chapter{Tensor fields} \label{s:ten}

\minitoc

\section{Introduction}

Having presented vector fields in Chap.~\ref{s:vec}, we move now to more
general tensor fields. We keep the same example manifold, $M = \mathbb{S}^2$,
as in Chap.~\ref{s:man} and \ref{s:vec}.

\section{Differential forms}

Let us continue with the same example notebook as that considered in
Chap.~\ref{s:vec}. There, we had introduced $f$ as a scalar field on
the 2-dimensional manifold $M = \mathbb{S}^2$ (cf.\ Sec.~\ref{s:vec:tangent_impl}).
The differential of $f$ is a 1-form on $M$:
\begin{NBin}
df = f.differential()
df
\end{NBin}
\begin{NBoutM}
\mathrm{d}f
\end{NBoutM}
\begin{NBin}
print(df)
\end{NBin}
\begin{NBprint}
1-form df on the 2-dimensional differentiable manifold M
\end{NBprint}
A 1-form is actually a tensor field of type $(0,1)$:
\begin{NBin}
df.tensor_type()
\end{NBin}
\begin{NBoutM}
\left(0, 1\right)
\end{NBoutM}
while a vector field is a tensor field of type $(1,0)$:
\begin{NBin}
v.tensor_type()
\end{NBin}
\begin{NBoutM}
\left(1, 0\right)
\end{NBoutM}
Specific 1-forms are those forming the dual basis (coframe) of a given vector
frame: for instance for the vector frame \code{eU} = $(\dert{}{x},\dert{}{y})$
on $U$, considered as a basis of the free $C^\infty(U)$-module $\mathfrak{X}(U)$,
we have:
\begin{NBin}
eU.dual_basis()
\end{NBin}
\begin{NBoutM}
\left(U, \left(\mathrm{d} x,\mathrm{d} y\right)\right)
\end{NBoutM}
\begin{NBin}
print(eU.dual_basis()[0])
\end{NBin}
\begin{NBprint}
1-form dx on the Open subset U of the 2-dimensional differentiable manifold M
\end{NBprint}
Since \code{eU} is the default frame on $M$, the default display of $\mathrm{d}f$
is performed in terms of \code{eU}'s coframe:
\begin{NBin}
df.display()
\end{NBin}
\begin{NBoutM}
\mathrm{d}f = \left( -\frac{2 \, x}{x^{4} + y^{4} + 2 \, {\left(x^{2} + 1\right)} y^{2} + 2 \, x^{2} + 1} \right) \mathrm{d} x + \left( -\frac{2 \, y}{x^{4} + y^{4} + 2 \, {\left(x^{2} + 1\right)} y^{2} + 2 \, x^{2} + 1} \right) \mathrm{d} y
\end{NBoutM}
We may check that in this basis, the components of $\left. \mathrm{d}f \right| _U$
are nothing but the partial derivatives of the coordinate expression of $f$
with respect to coordinates $(x,y)$:
\begin{NBin}
df[0] == diff(f.expr(), x)
\end{NBin}
\begin{NBout}
\texttt{True}
\end{NBout}
\begin{NBin}
df[1] == diff(f.expr(), y)
\end{NBin}
\begin{NBout}
\texttt{True}
\end{NBout}
In the coframe associated with \code{eV} = $(\dert{}{x'},\dert{}{y'})$:
\begin{NBin}
df.display(eV)
\end{NBin}
\begin{NBout}
$\displaystyle
\mathrm{d}f = \left( \frac{2 \, {x'}}{{x'}^{4} + {y'}^{4} + 2 \, {\left({x'}^{2} + 1\right)} {y'}^{2} + 2 \, {x'}^{2} + 1} \right) \mathrm{d} {x'} $\\
$\displaystyle
+ \left( \frac{2 \, {y'}}{{x'}^{4} + {y'}^{4} + 2 \, {\left({x'}^{2} + 1\right)} {y'}^{2} + 2 \, {x'}^{2} + 1} \right) \mathrm{d} {y'}$
\end{NBout}
Since \code{eV} is not the default vector frame on $M$ and \code{XV} = $(V,(x',y'))$
is not the default chart on $M$, we get the individual components by
specifying both \code{eV} and \code{XV}, in addition to the index, in the
square-bracket operator:
\begin{NBin}
df[eV,0,XV]
\end{NBin}
\begin{NBoutM}
\frac{2 \, {x'}}{{x'}^{4} + {y'}^{4} + 2 \, {\left({x'}^{2} + 1\right)} {y'}^{2} + 2 \, {x'}^{2} + 1}
\end{NBoutM}
We may then check that the components in the frame \code{eV}
are the partial derivatives with respect to the coordinates \code{xp} = $x'$ and
\code{yp} = $y'$ of the chart \code{XV}:
\begin{NBin}
df[eV,0,XV] == diff(f.expr(XV), xp)
\end{NBin}
\begin{NBout}
\texttt{True}
\end{NBout}
\begin{NBin}
df[eV,1,XV] == diff(f.expr(XV), yp)
\end{NBin}
\begin{NBout}
\texttt{True}
\end{NBout}
The parent of $\mathrm{d}f$ is the set $\Omega^1(M)$ of all 1-forms on $M$,
considered as a $C^\infty(M)$-module:
\begin{NBin}
print(df.parent())
df.parent()
\end{NBin}
\begin{NBprint}
Module Omega^1(M) of 1-forms on the 2-dimensional differentiable manifold M
\end{NBprint}
\begin{NBoutM}
\Omega^{1}\left(M\right)
\end{NBoutM}
\begin{NBin}
df.parent().base_ring()
\end{NBin}
\begin{NBoutM}
C^{\infty}\left(M\right)
\end{NBoutM}
This module is actually the dual of the vector-field module $\mathfrak{X}(M)$,
which is represented
by the Python object \code{YM} (cf.\ Sec.~\ref{s:vec:vector_field_impl}):
\begin{NBin}
YM.dual()
\end{NBin}
\begin{NBoutM}
\Omega^{1}\left(M\right)
\end{NBoutM}
Consequently, a 1-form acts on vector fields, yielding an element of
$C^\infty(M)$, i.e.\ a scalar field:
\begin{NBin}
print(df(v))
\end{NBin}
\begin{NBprint}
Scalar field df(v) on the 2-dimensional differentiable manifold M
\end{NBprint}
This scalar field is nothing but the result of the action of $\w{v}$ on $f$
discussed in Sec.~\ref{s:vec:action_on_scalar}:
\begin{NBin}
df(v) == v(f)
\end{NBin}
\begin{NBout}
\texttt{True}
\end{NBout}

\section{More general tensor fields}

We construct a tensor of type $(1,1)$ by taking the tensor product
$\w{v}\otimes \mathrm{d}f$:
\begin{NBin}
t = v * df
t
\end{NBin}
\begin{NBoutM}
\mbox{Tensor field of type (1,1) on the 2-dimensional differentiable manifold M}
\end{NBoutM}
\begin{NBin}
t.display()
\end{NBin}
\begin{NBoutM}
v\otimes \mathrm{d}f = \left( -\frac{2 \, x}{x^{6} + y^{6} + 3 \, {\left(x^{2} + 1\right)} y^{4} + 3 \, x^{4} + 3 \, {\left(x^{4} + 2 \, x^{2} + 1\right)} y^{2} + 3 \, x^{2} + 1} \right) \frac{\partial}{\partial x }\otimes \mathrm{d} x + \left( -\frac{2 \, y}{x^{6} + y^{6} + 3 \, {\left(x^{2} + 1\right)} y^{4} + 3 \, x^{4} + 3 \, {\left(x^{4} + 2 \, x^{2} + 1\right)} y^{2} + 3 \, x^{2} + 1} \right) \frac{\partial}{\partial x }\otimes \mathrm{d} y + \left( \frac{4 \, x}{x^{4} + y^{4} + 2 \, {\left(x^{2} + 1\right)} y^{2} + 2 \, x^{2} + 1} \right) \frac{\partial}{\partial y }\otimes \mathrm{d} x + \left( \frac{4 \, y}{x^{4} + y^{4} + 2 \, {\left(x^{2} + 1\right)} y^{2} + 2 \, x^{2} + 1} \right) \frac{\partial}{\partial y }\otimes \mathrm{d} y
\end{NBoutM}
\begin{NBin}
t.display(eV)
\end{NBin}
\begin{NBoutM}
v\otimes \mathrm{d}f = \left( -\frac{2 \, {\left({x'}^{5} - 4 \, {x'}^{2} {y'}^{3} - {x'} {y'}^{4} - 4 \, {\left({x'}^{4} + {x'}^{2}\right)} {y'}\right)}}{{x'}^{6} + {y'}^{6} + 3 \, {\left({x'}^{2} + 1\right)} {y'}^{4} + 3 \, {x'}^{4} + 3 \, {\left({x'}^{4} + 2 \, {x'}^{2} + 1\right)} {y'}^{2} + 3 \, {x'}^{2} + 1} \right) \frac{\partial}{\partial {x'} }\otimes \mathrm{d} {x'} + \left( -\frac{2 \, {\left({x'}^{4} {y'} - 4 \, {x'} {y'}^{4} - {y'}^{5} - 4 \, {\left({x'}^{3} + {x'}\right)} {y'}^{2}\right)}}{{x'}^{6} + {y'}^{6} + 3 \, {\left({x'}^{2} + 1\right)} {y'}^{4} + 3 \, {x'}^{4} + 3 \, {\left({x'}^{4} + 2 \, {x'}^{2} + 1\right)} {y'}^{2} + 3 \, {x'}^{2} + 1} \right) \frac{\partial}{\partial {x'} }\otimes \mathrm{d} {y'} + \left( -\frac{4 \, {\left({x'}^{5} + {x'}^{4} {y'} + {x'}^{2} {y'}^{3} - {x'} {y'}^{4} + {x'}^{3} - {x'} {y'}^{2}\right)}}{{x'}^{6} + {y'}^{6} + 3 \, {\left({x'}^{2} + 1\right)} {y'}^{4} + 3 \, {x'}^{4} + 3 \, {\left({x'}^{4} + 2 \, {x'}^{2} + 1\right)} {y'}^{2} + 3 \, {x'}^{2} + 1} \right) \frac{\partial}{\partial {y'} }\otimes \mathrm{d} {x'} + \left( -\frac{4 \, {\left({x'}^{3} {y'}^{2} + {x'} {y'}^{4} - {y'}^{5} - {y'}^{3} + {\left({x'}^{4} + {x'}^{2}\right)} {y'}\right)}}{{x'}^{6} + {y'}^{6} + 3 \, {\left({x'}^{2} + 1\right)} {y'}^{4} + 3 \, {x'}^{4} + 3 \, {\left({x'}^{4} + 2 \, {x'}^{2} + 1\right)} {y'}^{2} + 3 \, {x'}^{2} + 1} \right) \frac{\partial}{\partial {y'} }\otimes \mathrm{d} {y'}
\end{NBoutM}
We can use the method \code{display\_comp()} for a display component by
component:
\begin{NBin}
t.display_comp()
\end{NBin}
\begin{NBoutM}
\begin{array}{lcl} v\otimes \mathrm{d}f_{ \phantom{\, x} \, x }^{ \, x \phantom{\, x} } & = & -\frac{2 \, x}{x^{6} + y^{6} + 3 \, {\left(x^{2} + 1\right)} y^{4} + 3 \, x^{4} + 3 \, {\left(x^{4} + 2 \, x^{2} + 1\right)} y^{2} + 3 \, x^{2} + 1} \\ v\otimes \mathrm{d}f_{ \phantom{\, x} \, y }^{ \, x \phantom{\, y} } & = & -\frac{2 \, y}{x^{6} + y^{6} + 3 \, {\left(x^{2} + 1\right)} y^{4} + 3 \, x^{4} + 3 \, {\left(x^{4} + 2 \, x^{2} + 1\right)} y^{2} + 3 \, x^{2} + 1} \\ v\otimes \mathrm{d}f_{ \phantom{\, y} \, x }^{ \, y \phantom{\, x} } & = & \frac{4 \, x}{x^{4} + y^{4} + 2 \, {\left(x^{2} + 1\right)} y^{2} + 2 \, x^{2} + 1} \\ v\otimes \mathrm{d}f_{ \phantom{\, y} \, y }^{ \, y \phantom{\, y} } & = & \frac{4 \, y}{x^{4} + y^{4} + 2 \, {\left(x^{2} + 1\right)} y^{2} + 2 \, x^{2} + 1} \end{array}
\end{NBoutM}
The parent of $t$ is the set $\mathcal{T}^{(1,1)}(M)$ of all type-$(1,1)$
tensor fields on $M$,
considered as a $C^\infty(M)$-module:
\begin{NBin}
print(t.parent())
t.parent()
\end{NBin}
\begin{NBprint}
Module T^(1,1)(M) of type-(1,1) tensors fields on the 2-dimensional
differentiable manifold M
\end{NBprint}
\begin{NBoutM}
\mathcal{T}^{(1,1)}\left(M\right)
\end{NBoutM}
\begin{NBin}
t.parent().base_ring()
\end{NBin}
\begin{NBoutM}
C^{\infty}\left(M\right)
\end{NBoutM}

As for vector fields, since $M$ is not parallelizable, the $C^\infty(M)$-module
$\mathcal{T}^{(1,1)}(M)$ is not free and the tensor fields are described by
their restrictions to parallelizable subdomains:
\begin{NBin}
t._restrictions
\end{NBin}
\begin{NBoutM}
\left\{V : v\otimes \mathrm{d}f, U : v\otimes \mathrm{d}f\right\}
\end{NBoutM}
These restrictions form free modules:
\begin{NBin}
print(t._restrictions[U].parent())
\end{NBin}
\begin{NBprint}
Free module T^(1,1)(U) of type-(1,1) tensors fields on the Open subset U of
the 2-dimensional differentiable manifold M
\end{NBprint}
\begin{NBin}
t._restrictions[U].parent().base_ring()
\end{NBin}
\begin{NBoutM}
C^{\infty}\left(U\right)
\end{NBoutM}


\section{Riemannian metric}

\subsection{Defining a metric}

The standard metric on $M=\mathbb{S}^2$ is that induced by the Euclidean metric of $\mathbb{R}^3$. Let us start by defining the latter:
\begin{NBin}
h = R3.metric('h')
h[0,0], h[1,1], h[2, 2] = 1, 1, 1
h.display()
\end{NBin}
\begin{NBoutM}
h = \mathrm{d} X\otimes \mathrm{d} X+\mathrm{d} Y\otimes \mathrm{d} Y+\mathrm{d} Z\otimes \mathrm{d} Z
\end{NBoutM}
The metric $g$ on $M$ is the pullback of $h$ associated with the embedding $\Phi$
introduced in Sec.~\ref{s:vec:tangent_impl}:
\begin{NBin}
g = M.metric('g')
g.set( Phi.pullback(h) )
print(g)
\end{NBin}
\begin{NBprint}
Riemannian metric g on the 2-dimensional differentiable manifold M
\end{NBprint}
Note that we could have defined $g$ intrinsically, i.e.\ by providing its components in the two vector frames \code{eU} and \code{eV}, as we did for the metric $h$ on $\mathbb{R}^3$. Instead, we have chosen to get it as the pullback by $\Phi$ of $h$, as an example of pullback associated with some differential map.

The metric is a symmetric tensor field of type (0,2):
\begin{NBin}
g.tensor_type()
\end{NBin}
\begin{NBoutM}
\left(0, 2\right)
\end{NBoutM}
The expression of the metric in terms of the default frame on $M$ (\code{eU}):
\begin{NBin}
g.display()
\end{NBin}
\begin{NBout}
$\displaystyle
g = \left( \frac{4}{x^{4} + y^{4} + 2 \, {\left(x^{2} + 1\right)} y^{2} + 2 \, x^{2} + 1} \right) \mathrm{d} x\otimes \mathrm{d} x $\\
$\displaystyle
+ \left( \frac{4}{x^{4} + y^{4} + 2 \, {\left(x^{2} + 1\right)} y^{2} + 2 \, x^{2} + 1} \right) \mathrm{d} y\otimes \mathrm{d} y$
\end{NBout}
We may factorize the metric components to get a better display:
\begin{NBin}
g[0,0].factor(); g[1,1].factor()
\end{NBin}
\begin{NBoutM}
\frac{4}{{\left(x^{2} + y^{2} + 1\right)}^{2}}
\end{NBoutM}
\begin{NBin}
g.display()
\end{NBin}
\begin{NBoutM}
g = \frac{4}{{\left(x^{2} + y^{2} + 1\right)}^{2}} \mathrm{d} x\otimes \mathrm{d} x + \frac{4}{{\left(x^{2} + y^{2} + 1\right)}^{2}} \mathrm{d} y\otimes \mathrm{d} y
\end{NBoutM}
A matrix view of the components of $g$ in the manifold's default frame:
\begin{NBin}
g[:]
\end{NBin}
\begin{NBoutM}
\left(\begin{array}{rr}
    \frac{4}{{\left(x^{2} + y^{2} + 1\right)}^{2}} & 0 \\
    0 & \frac{4}{{\left(x^{2} + y^{2} + 1\right)}^{2}}
    \end{array}\right)
\end{NBoutM}
Display in terms of the vector frame $(V, (\partial_{x'}, \partial_{y'}))$:
\begin{NBin}
g.display(eV)
\end{NBin}
\begin{NBout}
$\displaystyle
g = \left( \frac{4}{{x'}^{4} + {y'}^{4} + 2 \, {\left({x'}^{2} + 1\right)} {y'}^{2} + 2 \, {x'}^{2} + 1} \right) \mathrm{d} {x'}\otimes \mathrm{d} {x'}$\\
$\displaystyle + \left( \frac{4}{{x'}^{4} + {y'}^{4} + 2 \, {\left({x'}^{2} + 1\right)} {y'}^{2} + 2 \, {x'}^{2} + 1} \right) \mathrm{d} {y'}\otimes \mathrm{d} {y'}$
\end{NBout}
The metric acts on vector field pairs, resulting in a scalar field:
\begin{NBin}
print(g(v,v))
\end{NBin}
\begin{NBprint}
Scalar field g(v,v) on the 2-dimensional differentiable manifold M
\end{NBprint}
\begin{NBin}
g(v,v).parent()
\end{NBin}
\begin{NBoutM}
C^{\infty}\left(M\right)
\end{NBoutM}
\begin{NBin}
g(v,v).display()
\end{NBin}
\begin{NBoutM}
\begin{array}{llcl} g\left(v,v\right):& M & \longrightarrow & \mathbb{R} \\ \mbox{on}\ U : & \left(x, y\right) & \longmapsto & \frac{4 \, {\left(4 \, x^{4} + 4 \, y^{4} + 8 \, {\left(x^{2} + 1\right)} y^{2} + 8 \, x^{2} + 5\right)}}{x^{8} + y^{8} + 4 \, {\left(x^{2} + 1\right)} y^{6} + 4 \, x^{6} + 6 \, {\left(x^{4} + 2 \, x^{2} + 1\right)} y^{4} + 6 \, x^{4} + 4 \, {\left(x^{6} + 3 \, x^{4} + 3 \, x^{2} + 1\right)} y^{2} + 4 \, x^{2} + 1} \\[1ex] \mbox{on}\ V : & \left({x'}, {y'}\right) & \longmapsto & \frac{4 \, {\left(5 \, {x'}^{8} + 5 \, {y'}^{8} + 4 \, {\left(5 \, {x'}^{2} + 2\right)} {y'}^{6} + 8 \, {x'}^{6} + 2 \, {\left(15 \, {x'}^{4} + 12 \, {x'}^{2} + 2\right)} {y'}^{4} + 4 \, {x'}^{4} + 4 \, {\left(5 \, {x'}^{6} + 6 \, {x'}^{4} + 2 \, {x'}^{2}\right)} {y'}^{2}\right)}}{{x'}^{8} + {y'}^{8} + 4 \, {\left({x'}^{2} + 1\right)} {y'}^{6} + 4 \, {x'}^{6} + 6 \, {\left({x'}^{4} + 2 \, {x'}^{2} + 1\right)} {y'}^{4} + 6 \, {x'}^{4} + 4 \, {\left({x'}^{6} + 3 \, {x'}^{4} + 3 \, {x'}^{2} + 1\right)} {y'}^{2} + 4 \, {x'}^{2} + 1} \end{array}
\end{NBoutM}

\subsection{Levi-Civita connection}

The Levi-Civita connection associated with the metric $g$ is
\begin{NBin}
nab = g.connection()
print(nab)
nab
\end{NBin}
\begin{NBprint}
Levi-Civita connection nabla_g associated with the Riemannian metric g on
the 2-dimensional differentiable manifold M
\end{NBprint}
\begin{NBoutM}
\nabla_{g}
\end{NBoutM}
The nonzero Christoffel symbols of $g$ (skipping those that can be deduced by symmetry on the last two indices) w.r.t. the chart \code{XU}:
\begin{NBin}
g.christoffel_symbols_display(chart=XU)
\end{NBin}
\begin{NBoutM}
\begin{array}{lcl} \Gamma_{ \phantom{\, x} \, x \, x }^{ \, x \phantom{\, x} \phantom{\, x} } & = & -\frac{2 \, x}{x^{2} + y^{2} + 1} \\ \Gamma_{ \phantom{\, x} \, x \, y }^{ \, x \phantom{\, x} \phantom{\, y} } & = & -\frac{2 \, y}{x^{2} + y^{2} + 1} \\ \Gamma_{ \phantom{\, x} \, y \, y }^{ \, x \phantom{\, y} \phantom{\, y} } & = & \frac{2 \, x}{x^{2} + y^{2} + 1} \\ \Gamma_{ \phantom{\, y} \, x \, x }^{ \, y \phantom{\, x} \phantom{\, x} } & = & \frac{2 \, y}{x^{2} + y^{2} + 1} \\ \Gamma_{ \phantom{\, y} \, x \, y }^{ \, y \phantom{\, x} \phantom{\, y} } & = & -\frac{2 \, x}{x^{2} + y^{2} + 1} \\ \Gamma_{ \phantom{\, y} \, y \, y }^{ \, y \phantom{\, y} \phantom{\, y} } & = & -\frac{2 \, y}{x^{2} + y^{2} + 1} \end{array}
\end{NBoutM}
$\nabla_g$ acting on the vector field $\w{v}$:
\begin{NBin}
Dv = nab(v)
print(Dv)
\end{NBin}
\begin{NBprint}
Tensor field nabla_g(v) of type (1,1) on the 2-dimensional differentiable
manifold M
\end{NBprint}
\begin{NBin}
Dv.display()
\end{NBin}
\begin{NBout}
$\displaystyle
\nabla_{g} v = \left( \frac{4 \, {\left(y^{3} + {\left(x^{2} + 1\right)} y - x\right)}}{x^{4} + y^{4} + 2 \, {\left(x^{2} + 1\right)} y^{2} + 2 \, x^{2} + 1} \right) \frac{\partial}{\partial x }\otimes \mathrm{d} x $\\
$\displaystyle
+ \left( -\frac{4 \, {\left(x^{3} + x y^{2} + x + y\right)}}{x^{4} + y^{4} + 2 \, {\left(x^{2} + 1\right)} y^{2} + 2 \, x^{2} + 1} \right) \frac{\partial}{\partial x }\otimes \mathrm{d} y $\\
$\displaystyle
+ \left( \frac{2 \, {\left(2 \, x^{3} + 2 \, x y^{2} + 2 \, x + y\right)}}{x^{4} + y^{4} + 2 \, {\left(x^{2} + 1\right)} y^{2} + 2 \, x^{2} + 1} \right) \frac{\partial}{\partial y }\otimes \mathrm{d} x$\\
$\displaystyle
 + \left( \frac{2 \, {\left(2 \, y^{3} + 2 \, {\left(x^{2} + 1\right)} y - x\right)}}{x^{4} + y^{4} + 2 \, {\left(x^{2} + 1\right)} y^{2} + 2 \, x^{2} + 1} \right) \frac{\partial}{\partial y }\otimes \mathrm{d} y$
\end{NBout}

\subsection{Curvature}

The Riemann curvature tensor of the metric $g$ is
\begin{NBin}
Riem = g.riemann()
print(Riem)
Riem.display()
\end{NBin}
\begin{NBprint}
Tensor field Riem(g) of type (1,3) on the 2-dimensional differentiable
manifold M
\end{NBprint}
\begin{NBout}
$\displaystyle
\mathrm{Riem}\left(g\right) = \left( \frac{4}{x^{4} + y^{4} + 2 \, {\left(x^{2} + 1\right)} y^{2} + 2 \, x^{2} + 1} \right) \frac{\partial}{\partial x }\otimes \mathrm{d} y\otimes \mathrm{d} x\otimes \mathrm{d} y $\\
$\displaystyle + \left( -\frac{4}{x^{4} + y^{4} + 2 \, {\left(x^{2} + 1\right)} y^{2} + 2 \, x^{2} + 1} \right) \frac{\partial}{\partial x }\otimes \mathrm{d} y\otimes \mathrm{d} y\otimes \mathrm{d} x$\\
$\displaystyle
 + \left( -\frac{4}{x^{4} + y^{4} + 2 \, {\left(x^{2} + 1\right)} y^{2} + 2 \, x^{2} + 1} \right) \frac{\partial}{\partial y }\otimes \mathrm{d} x\otimes \mathrm{d} x\otimes \mathrm{d} y$\\
$\displaystyle
 + \left( \frac{4}{x^{4} + y^{4} + 2 \, {\left(x^{2} + 1\right)} y^{2} + 2 \, x^{2} + 1} \right) \frac{\partial}{\partial y }\otimes \mathrm{d} x\otimes \mathrm{d} y\otimes \mathrm{d} x$
\end{NBout}
The components of the Riemann tensor in the default frame on $M$ are
\begin{NBin}
Riem.display_comp()
\end{NBin}
\begin{NBoutM}
\begin{array}{lcl} \mathrm{Riem}\left(g\right)_{ \phantom{\, x} \, y \, x \, y }^{ \, x \phantom{\, y} \phantom{\, x} \phantom{\, y} } & = & \frac{4}{x^{4} + y^{4} + 2 \, {\left(x^{2} + 1\right)} y^{2} + 2 \, x^{2} + 1} \\ \mathrm{Riem}\left(g\right)_{ \phantom{\, x} \, y \, y \, x }^{ \, x \phantom{\, y} \phantom{\, y} \phantom{\, x} } & = & -\frac{4}{x^{4} + y^{4} + 2 \, {\left(x^{2} + 1\right)} y^{2} + 2 \, x^{2} + 1} \\ \mathrm{Riem}\left(g\right)_{ \phantom{\, y} \, x \, x \, y }^{ \, y \phantom{\, x} \phantom{\, x} \phantom{\, y} } & = & -\frac{4}{x^{4} + y^{4} + 2 \, {\left(x^{2} + 1\right)} y^{2} + 2 \, x^{2} + 1} \\ \mathrm{Riem}\left(g\right)_{ \phantom{\, y} \, x \, y \, x }^{ \, y \phantom{\, x} \phantom{\, y} \phantom{\, x} } & = & \frac{4}{x^{4} + y^{4} + 2 \, {\left(x^{2} + 1\right)} y^{2} + 2 \, x^{2} + 1} \end{array}
\end{NBoutM}
The parent of the Riemann tensor is the $C^\infty(M)$-module of
type-(1,3) tensor fields on $M$:
\begin{NBin}
print(Riem.parent())
\end{NBin}
\begin{NBprint}
Module T^(1,3)(M) of type-(1,3) tensors fields on the 2-dimensional
differentiable manifold M
\end{NBprint}
The Riemann tensor is antisymmetric on its two last indices (i.e.\ the indices
at position 2 and 3, the first index being at position 0):
\begin{NBin}
Riem.symmetries()
\end{NBin}
\begin{NBout}
\texttt{no symmetry; antisymmetry:~(2, 3)}
\end{NBout}
The Riemann tensor of the Euclidean metric $h$ on $\mathbb{R}^3$ is identically zero,
i.e.\ $h$ is a flat metric:
\begin{NBin}
h.riemann().display()
\end{NBin}
\begin{NBoutM}
\mathrm{Riem}\left(h\right) = 0
\end{NBoutM}
The Ricci tensor is
\begin{NBin}
Ric = g.ricci()
Ric.display()
\end{NBin}
\begin{NBout}
$\displaystyle
\mathrm{Ric}\left(g\right) = \left( \frac{4}{x^{4} + y^{4} + 2 \, {\left(x^{2} + 1\right)} y^{2} + 2 \, x^{2} + 1} \right) \mathrm{d} x\otimes \mathrm{d} x $\\
$\displaystyle
+ \left( \frac{4}{x^{4} + y^{4} + 2 \, {\left(x^{2} + 1\right)} y^{2} + 2 \, x^{2} + 1} \right) \mathrm{d} y\otimes \mathrm{d} y$
\end{NBout}
while the Ricci scalar is
\begin{NBin}
R = g.ricci_scalar()
R.display()
\end{NBin}
\begin{NBoutM}
\begin{array}{llcl} \mathrm{r}\left(g\right):& M & \longrightarrow & \mathbb{R} \\ \mbox{on}\ U : & \left(x, y\right) & \longmapsto & 2 \\ \mbox{on}\ V : & \left({x'}, {y'}\right) & \longmapsto & 2 \end{array}
\end{NBoutM}
We recover the fact that $(\mathbb{S}^2,g)$ is a Riemannian manifold of constant positive curvature.

In dimension 2, the Riemann curvature tensor is entirely determined by the Ricci scalar $R$ according to
\be
 R^i_{\ \, jlk} = \frac{R}{2} \left( \delta^i_{\ \, k} g_{jl} - \delta^i_{\ \, l} g_{jk} \right)
\ee
Let us check this formula here, under the form
$R^i_{\ \, jlk} = -R g_{j[k} \delta^i_{\ \, l]}$:
\begin{NBin}
delta = M.tangent_identity_field()
Riem == - R*(g*delta).antisymmetrize(2,3)
\end{NBin}
\begin{NBout}
\texttt{True}
\end{NBout}
Similarly the relation $\mathrm{Ric} = (R/2)\; g$ must hold:
\begin{NBin}
Ric == (R/2)*g
\end{NBin}
\begin{NBout}
\texttt{True}
\end{NBout}

\subsection{Volume form}

The \defin{volume form} (or \defin{Levi-Civita tensor}) associated with the
metric $g$ and for which the vector frame $(\partial_x,\partial_y)$ is
right-handed is the following 2-form:
\begin{NBin}
eps = g.volume_form()
print(eps)
eps.display()
\end{NBin}
\begin{NBoutM}
\epsilon_{g} = \left( \frac{4}{x^{4} + y^{4} + 2 \, {\left(x^{2} + 1\right)} y^{2} + 2 \, x^{2} + 1} \right) \mathrm{d} x\wedge \mathrm{d} y
\end{NBoutM}
The exterior derivative of $\epsilon_g$ is a 3-form:
\begin{NBin}
print(eps.exterior_derivative())
\end{NBin}
\begin{NBprint}
3-form deps_g on the 2-dimensional differentiable manifold M
\end{NBprint}
Of course, since the dimension of $M$ is 2, all 3-forms vanish identically:
\begin{NBin}
eps.exterior_derivative().display()
\end{NBin}
\begin{NBoutM}
\mathrm{d}\epsilon_{g} = 0
\end{NBoutM}
