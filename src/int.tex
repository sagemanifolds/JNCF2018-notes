\chapter{Introduction} \label{s:int}

\minitoc

\section{What is tensor calculus on manifolds?}

While we shall provide precise definitions of the involved mathematical
objects in Chaps.~\ref{s:man} and \ref{s:vec},
we outline here briefly what is meant by
\emph{tensor calculus}. Basically, this is calculus on vector fields,
and more generally tensor fields, on differentiable manifolds, involving the following operations \cite{Lee13}:
\begin{itemize}
\item arithmetics of tensor fields;
\item tensor product, contraction;
\item (anti)symmetrization;
\item Lie derivative along vector fields;
\item pullback and pushforward associated with smooth manifold maps;
\item exterior calculus on differential forms;
\item affine connections (torsion, curvature).
\end{itemize}
Moreover, on pseudo-Riemannian manifolds, i.e. differentiable manifolds endowed
with a metric tensor, we may add the following
operations \cite{Lee97,ONeil83}:
\begin{itemize}
\item musical isomorphisms (i.e. raising and lowering indices with the metric tensor);
\item covariant derivatives with the Levi-Civita connection;
\item evaluating the curvature tensor of the Levi-Civita connection (Riemann tensor);
\item Hodge duality;
\item computing geodesics.
\end{itemize}


\section{A few words of history}

Symbolic tensor calculus has a long history, which started
almost as soon as computer algebra itself in the 1960s.
Probably, the first tensor calculus program was \soft{GEOM}, written by J.G.~Fletcher
in 1965 \cite{Fletc67}. Its main capability was to compute the Riemann tensor
of a given metric. In 1969, R.A. d'Inverno developed \soft{ALAM}
(for \emph{Atlas Lisp Algebraic Manipulator}) and used it to compute
the Riemann and Ricci tensors of the Bondi metric.
According to \cite{Skea94},
the original calculations took Bondi and collaborators 6 months to finish,
while the computation with \soft{ALAM} took 4 minutes and yielded the
discovery of 6 errors in the original paper by Bondi et al.
Since then, numerous packages have been developed; the reader is referred to \cite{MacCa02}
for a review of computer algebra
systems for general relativity prior to 2002, and to \cite{KorolKS13} for a more recent review
focused on tensor calculus.
It is also worth to point out the extensive list of
tensor calculus packages maintained by J. M. Martin-Garcia at
\url{http://www.xact.es/links.html}.

%%%%%%%%%%%%%%%%%%%%%%%%%%%%%%

\section{Software for differential geometry}

Software packages for differential geometry and tensor calculus can be
classified in two categories:
\begin{enumerate}
\item Applications atop some general purpose computer algebra system.
Notable examples\footnote{See \url{https://en.wikipedia.org/wiki/Tensor_software}
for more examples.} are
the \soft{xAct} suite \cite{Marti08} and \soft{Ricci} \cite{ricci}, both
running atop \soft{Mathematica},
\soft{DifferentialGeometry} \cite{AnderT12} integrated into \soft{Maple},
\soft{GRTensorIII} \cite{grtensorIII} atop \soft{Maple}, \soft{Atlas 2} \cite{atlas2}
for \soft{Mathematica} and \soft{Maple}, and
\soft{SageManifolds} \cite{sagemanifolds} integrated in \soft{SageMath}.
\item Standalone applications. Recent examples are \soft{Cadabra}
\cite{Peete07} (field theory),
\soft{SnapPy} \cite{snappy} (topology and geometry of 3-manifolds)  and
\soft{Redberry} \cite{BolotP13} (tensors); older examples can be found in
Ref.~\cite{MacCa02}.
\end{enumerate}
All applications listed in the second category are free software. In
the first category, \soft{xAct} and \soft{Ricci} are also free software, but
they require a proprietary product, the source code of which is closed (\soft{Mathematica}).

As far as tensor calculus is concerned, the above packages can be distinguished by
the type of computation that they perform:
abstract calculus (\soft{xAct/xTensor}, \soft{Ricci}, \soft{Cadabra}, \soft{Redberry}),
or component calculus (\soft{xAct/xCoba}, \soft{DifferentialGeometry}, \soft{GRTensorIII},
\soft{Atlas 2}, \soft{SageManifolds}).
In the first category, tensor operations such as contraction or covariant differentiation
are performed by manipulating the indices themselves rather than the components
to which they correspond. In the second category, vector frames are explicitly
introduced on the manifold and tensor operations are carried out on the components
in a given frame.


%%%%%%%%%%%%%%%%%%%%%%%%%%%%%%

\section{A brief overview of SageMath} \label{s:int:overview_Sage}

Since the tensor calculus method presented here is implemented in \Sage{}, we
give first a brief overview of it.

\Sage{}\footnote{\url{http://www.sagemath.org}} is a free, open-source mathematics software system, which is
based on the Python programming language. It makes use of over 90 open-source packages,
among which are \soft{Maxima}, \soft{Pynac} and \soft{SymPy} (symbolic calculations),
\soft{GAP} (group theory),
\soft{PARI/GP} (number theory), \soft{Singular} (polynomial computations),
\soft{matplotlib} (high quality 2D figures), and \soft{Jupyter} (graphical interface).
\Sage{} provides a uniform Python interface to all these packages; however,
\Sage{} is much more than a mere interface: it contains a large and increasing part of
original code (more than 750,000 lines of Python and Cython, involving 5344 classes).
\Sage{} was created in 2005 by W. Stein \cite{SteinJ05} and since
then its development has been sustained by more than a hundred researchers
(mostly mathematicians). Very good introductory textbooks about \Sage{} are
\cite{JoyneS14,Zimme13,Zimme18,Bard15}.

Apart from the syntax, which is based on a popular programming language
(Python) and not some custom script
language, a difference between \Sage{} and, e.g., \soft{Maple} or \soft{Mathematica}
is the use of the \emph{parent/element pattern}\index{parent}\index{element}. This framework more closely
reflects actual mathematics.
For instance, in \soft{Mathematica}, all objects
are trees of symbols and the program is essentially a set of
sophisticated rules to manipulate symbols. On the contrary, in \Sage{}
each object has a given type (i.e. is an instance of a given
Python class\footnote{Let us
recall that within an object-oriented programming language (as Python),
a \defin{class} is a structure to declare and store the
properties common to a set of objects. These properties
are data (called
\defin{attributes} or \defin{state variables}) and functions acting
on the data (called \defin{methods}). A specific realization of an object
within a given class is called an \defin{instance} of that class.}),
and one distinguishes \defin{parent} types, which model mathematical
sets with some structure (e.g. algebraic structure), from \defin{element} types,
which model set elements. Moreover, each parent belongs to some
dynamically generated class that encodes information
about its \emph{category}, in the mathematical sense of the word\footnote{See
\url{http://doc.sagemath.org/html/en/reference/categories/sage/categories/primer.html}
for a discussion of \Sage{}'s category framework}.
Automatic conversion rules, called \defin{coercions}\index{coercion},
prior to a binary operation, e.g. $x+y$ with $x$ and $y$ having different
parents, are implemented.

\section{The purpose of this lecture}

This lecture aims at presenting
a \emph{symbolic tensor calculus method} that
\begin{itemize}
\item runs on fully specified smooth manifolds (described by an atlas);
\item is not limited to a single coordinate chart or vector frame;
\item runs even on non-parallelizable manifolds (i.e. manifolds that cannot
be covered by a single vector frame);
\item is independent of the symbolic engine (e.g.  \soft{Pynac/Maxima},
\soft{SymPy},...) used to perform calculus at the level of coordinate expressions.
\end{itemize}
The aim is to present not only the main ideas of the method, but also some details of its implementation in \Sage{}. This implementation has been
performed via the \soft{SageManifolds} project:
\begin{center}
\url{http://sagemanifolds.obspm.fr},
\end{center}
the full list of contributors being available at
\begin{center}
\url{http://sagemanifolds.obspm.fr/authors.html}.
\end{center}
